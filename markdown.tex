\documentclass{beamer}
\usepackage[croatian]{babel}
\usepackage[utf8]{inputenc}
\usepackage{beamerthemeBerlin}
\usepackage{graphicx}
\usepackage{verbatim}
\usepackage{hyperref}
\begin{document}


\title{Markdown}
\author{Deni Klen, Fran Grenko i Mihael Petranović}
\institute{Tehnički fakultet Rijeka, Sveučilište u Rijeci}
\maketitle






\begin{frame}
	\frametitle{Što je Markdown?}

 	\begin{minipage}[0.2\textheight]{\textwidth}
 	\begin{columns}[T]
 	\begin{column}{0.8\textwidth}
 	\begin{itemize}
		\item{Markdown je lagani markup jezik sa sintaksom za oblikovanje običnog teksta čija je glavna značajka čitljivost}
		\item{Stvorili su ga John Gruber i Aaron Swartz 2004. godine}
		\item{Postoji više inačica koje se i dalje razvijaju \\ npr. \textit{CommonMark, GFM, Markdown Extra} ...}
	\end{itemize}
	\end{column}
	\begin{column}{0.2\textwidth}
	\includegraphics[width=2.5cm]{Slike/markdown.png}
	\end{column}
	\end{columns}
	\end{minipage}

\end{frame}





\begin{frame}
	\frametitle{Upotreba Markdown-a}

 	\begin{minipage}[0.2\textheight]{\textwidth}
 	\begin{columns}[T]
 	\begin{column}{0.8\textwidth}
 	\begin{itemize}
		\item{Osnovna upotreba je formatiranje, pisanje i uređivanje različiti tekstualnih datoteka, uljepšavanje teksta itd.}
		\item{\textit{*Ukošavanje*}, \textbf{**podebljavanje**}, \emph{isticanje} i \underline{podcrtavanje} teksta neke su od mogućnosti uljepšavanja teksta u MarkDownu}
		\item{Markdown također omogućuje kreiranje zaglavlja i lista te ubacivanje poveznica, slika i citata}
	\end{itemize}
	\end{column}
	\end{columns}
	\end{minipage}

\end{frame}

\begin{frame}[fragile]
	\frametitle{Primjer sintakse}

	\begin{verbatim}
	# Markdown
	Uz pomoć mardown-a možemo vrlo lako **podebljati** ,
	 *ukositi* i _podcrtati_ tekst.
	\end{verbatim}

	\begin{figure}[b]
		\includegraphics[width = 0.8\linewidth]{Slike/kod.png}
		\caption{Ispis}
	\end{figure}

\end{frame}





\begin{frame}
	\frametitle{CommonMark}

 	\begin{minipage}[0.2\textheight]{\textwidth}
 	\begin{columns}[T]
 	\begin{column}{0.8\textwidth}
 	\begin{itemize}
		\item{Jedna od najpopularnijih inačica prvotnog Markdown jezika}
		\item{Želja developera CommonMark-a je stvorit jedinstveni oblik Markdown sintakse koju bi svi mogli koristiti kako ne bi dolazilo do razlika u prikazivanju \\(razlike se događaju jer se tekst drugačije "čita" i prikazuje za različite inačice Markdown sintaksa)}
	\end{itemize}
	\end{column}
	\end{columns}
	\end{minipage}

\end{frame}





\begin{frame}
	\frametitle{GFM - GitHub Flavored Markdown}

 	\begin{minipage}[0.2\textheight]{\textwidth}
 	\begin{columns}[T]
 	\begin{column}{0.8\textwidth}
 	\begin{itemize}
		\item{Inačica koja se bazira ponajviše na CommonMark specifikacijama}
		\item{Koristi se na stranicama GitHub Enterprise i \href{https://github.com/}{GitHub.com}}
	\end{itemize}
	\end{column}
	\begin{column}{0.2\textwidth}
	\includegraphics[width=2.5cm]{Slike/githublogo.png}
	\end{column}
	\end{columns}
	\end{minipage}

\end{frame}





\begin{frame}
	\frametitle{Markdown Extra}

 	\begin{minipage}[0.2\textheight]{\textwidth}
 	\begin{columns}[T]
 	\begin{column}{0.8\textwidth}
 	\begin{itemize}
 		\item{Ekstenzija PHP Markdown-a}
		\item{Omogućuje upotrebu značajki koje se nisu mogle upotrebljavati u prvotnoj Markdown sintaksi}
		\item{Primjeri: 
		\begin{itemize}
			\item{Markdown unutar HTML blokova}
			\item{Fusnote}
			\item{Liste definicija}
			\item{Tablice}
			\item{Skraćenice}...
		\end{itemize}}
	\end{itemize}
	\end{column}
	\end{columns}
	\end{minipage}

\end{frame}

\begin{frame}
	\frametitle{Markdown Extra}

 	\begin{minipage}[0.2\textheight]{\textwidth}
 	\begin{columns}[T]
 	\begin{column}{0.8\textwidth}
 	\begin{itemize}
 		\item{Sources:
 		\begin{itemize}
 			\item{\url{https://en.wikipedia.org/wiki/Markdown}}
 			\item{\url{https://www.markdowntutorial.com/}}
 			\item{\url{https://commonmark.org/help/}}
 			\item{\url{https://www.markdownguide.org/}}
 			\item{\url{https://github.github.com/gfm/}}
 			\item{\url{https://michelf.ca/projects/php-markdown/extra/}}
 		\end{itemize}}
	\end{itemize}
	\end{column}
	\end{columns}
	\end{minipage}

\end{frame}



\end{document}
